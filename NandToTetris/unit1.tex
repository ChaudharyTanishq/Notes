\chapter{Boolean Functions and Gate Logic}

\section{Boolean Logic}

$x\ AND\ y \rightarrow x \land y$\\
$x\ OR\ y \rightarrow x \lor y$\\
$NOT(x) \rightarrow \neg x$\\

\subsection{Boolean Functions}

$f(x, y, z) = (x \land y) \lor (\neg x \land z)$

\begin{table}[h!]
    \begin{center} 
        \caption{$f(x, y, z)$}
        \label{tab:fxyz}
        \begin{tabular}{l|l|l|l}
            $x$ & $y$ & $z$ & $f$\\
            \hline
            0 & 0 & 0 & 0\\
            0 & 0 & 1 & 1\\
            0 & 1 & 0 & 0\\
            0 & 1 & 1 & 1\\
            1 & 0 & 0 & 0\\
            1 & 0 & 1 & 0\\
            1 & 1 & 0 & 1\\
            1 & 1 & 1 & 1\\
        \end{tabular}
    \end{center}
\end{table}

Both are identical representations

\subsection{Boolean Identies}

Commutative Laws:\\

$x \land y = y \land x$\\
$x \lor y = y \lor x$\\

Associative Laws:\\

$x \land (y \land z) = (x \land y) \land z$\\
$x \lor (y \lor z) = (x \lor y) \lor z$\\

Distributive Laws:\\

$x \land (y \lor z) = (x \land y) \lor (x \land z)$\\
$x \lor (y \land z) = (x \lor y) \land (x \lor z)$\\

De Morgan Laws:\\

$\neg(x \land y) = \neg(x) \lor \neg(y)$\\
$\neg(x \lor y) = \neg(x) \land \neg(y)$\\

\subsection{Boolean Functions Synthesis}

Given a Truth Table, how do we construct a boolean function for it?\\

Lets take an example, 

\begin{table}[h!]
    \begin{center} 
        \caption{Truth Table for a Boolean Function}
        \label{tab:example1}
        \begin{tabular}{l|l|l|l}
            $x$ & $y$ & $z$ & $f$\\
            \hline
            0 & 0 & 0 & 0\\
            0 & 0 & 1 & 1\\
            0 & 1 & 0 & 0\\
            0 & 1 & 1 & 0\\
            1 & 0 & 0 & 0\\
            1 & 0 & 1 & 0\\
            1 & 1 & 0 & 1\\
            1 & 1 & 1 & 0\\
        \end{tabular}
    \end{center}
\end{table}

This table can be represented by taking OR of the "true"" statements. 
We can represent the "true" statements by a boolean expression.
For example, $x = 0, y = 0, z = 1, f = 1$ can be represented by
$\neg x \land \neg y \land z$. This expression is only true when
$x = 0, y = 0, z = 1$ and false on all other cases. So, we can
represent every boolean function by AND of such terms.\\

The truth table \ref{tab:example1} can be represented by the expression:

$$
(\neg x \land \neg y \land z) \lor (x \land y \land \neg z)
$$

But what is the minimum size expression we can build from a truth
table? This is a NP-Complete problem and cannot be solved in
polynomial time if $P \neq NP$.\\

\subsection{Why NAND?}
We can represent every boolean expression only using NAND. This
can be trivially proved.

\section{Gate Logic}

A technique for implementing Boolean functions using logic gates.\\

Logic Gates:
\begin{enumerate}
    \item Elementary (Nand, And, Or, Not)
    \item Composite (Mux, Adder)
\end{enumerate}
