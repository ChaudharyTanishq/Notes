\documentclass[11pt]{article}   
\usepackage{fullpage}
\usepackage{amsfonts}
\usepackage{xcolor}

% Packages seperate from the og tex file
\usepackage{amssymb}
\usepackage{amsmath}
\usepackage{amsthm}
\usepackage{algorithm}
\usepackage{algorithmic}

% Theorem macros
\newtheorem{theorem}{Theorem}
\newtheorem{claim}[theorem]{Claim}
\newtheorem{proposition}[theorem]{Proposition}
\newtheorem{lemma}[theorem]{Lemma}
\newtheorem{corollary}[theorem]{Corollary}
\newtheorem{conjecture}[theorem]{Conjecture}
\newtheorem*{observation}{Observation}
\newtheorem*{example}{Example}
\newtheorem*{remark}{Remark}

% New macros
\newcommand{\R}{\ensuremath{\mathbb{R}}}
\newcommand{\N}{\ensuremath{\mathbb{N}}}

\newcommand{\F}{\mathbb{F}}
\newcommand{\np}{\mathop{\rm NP}}
\newcommand{\Z}{{\mathbb Z}}
\newcommand{\vol}{\mathop{\rm Vol}}
\newcommand{\conp}{\mathop{\rm co-NP}}
\newcommand{\atisp}{\mathop{\rm ATISP}}
\renewcommand{\vec}[1]{{\mathbf #1}}
\newcommand{\cupdot}{\mathbin{\mathaccent\cdot\cup}}
\newcommand{\mmod}[1]{\ (\mathrm{mod}\ #1)}  

\setlength{\parskip}{\medskipamount}
\setlength{\parindent}{0in}
%\input{dansmacs}


\begin{document}

\section*{Modern Complexity Theory Homework Zero}\label{homework-zero}

The aim of problem set is to help you to test, and if needed to brush up, up on the mathematical
background needed to be successful in this course. 

\begin{itemize}
    \item
        {\bf Collaboration:} You can collaborate with other students that are currently enrolled in
        this course  in brainstorming and thinking through approaches to
        solutions but you should write the solutions on your own and cannot
        share them with other students. 
    \item
        {\bf Owning your solution:} Always make sure that you ``own'' your solutions to this other problem
        sets. That is, you should always first grapple with the problems on
        your own, and even if you participate in brainstorming sessions, make
        sure that you completely understand the ideas and details underlying
        the solution. This is in your interest as it ensures you have a solid
        understanding of the course material, and will help in the midterms
        and final. Getting 80\% of the problem
        set questions right on your own will be much better to both your
        understanding than getting 100\% of the questions through
        gathering hints from others without true understanding.
    \item
        {\bf Serious violations:} Sharing questions or solutions with anyone outside this course,
        including posting on outside websites, is a violation of the honor
        code policy. Collaborating with anyone except students currently
        taking this course or using material from past years from this or
        other courses is a violation of the honor code policy.
    \item
        {\bf Submission Format:} The submitted PDF should be typed and in the same format and
        pagination as ours. Please include the text of the problems and write
        \textbf{Solution X:} before your solution. Please mark in gradescope 
        the pages where
        the solution to each question appears. Points will be deducted if you
        submit in a different format.
\end{itemize}

\textbf{By writing my name here I affirm that I am aware of all policies
and abided by them while working on this problem set:}

\textbf{Your name:} Kunwar Shaanjeet Singh Grover, Email: kunwar.shaanjeet@students.iiit.ac.in

\textbf{Collaborators:} (List here names of anyone you discussed
problems or ideas for solutions with)


\newpage


\subsection*{Questions}\label{questions}

Please solve the following problems. Some of these might be harder than
the others, so don't despair if they require more time to think or you
can't do them all. Just do your best. Also, you should only attempt the
bonus questions if you have the time to do so. If you don't have a proof
for a certain statement, be upfront about it. You can always explain
clearly what you are able to prove and the point at which you were
stuck. Also, for a non bonus question, you can always simply write
\textbf{``I don't know''} and you will get 15 percent of the credit for
this problem.

The discussion board for this course will be active even before the course
starts. If you are stuck on this problem set, you can use this discussion board to send
a private message to all instructors under the \texttt{e-office-hours}
folder.

This problem set has a total of \textbf{50 points} and \textbf{11 bonus
points}. A grade of 50 or more on this problem set is considered a
perfect grade. If you get stuck in any questions, you might find the
resources in the CS 121 background page at {\tt https://cs121.boazbarak.org/background/} helpful.

\textbf{Problem 0 (5 points):} Read fully the \textbf{Mathematical
Background Chapter} from the textbook at 
{\tt https://introtcs.org/public/lec\_00\_\_math\_background.pdf}.
This is probably the most important exercise in this problem set!!

\textbf{Solution 0:} I certify that I fully read the
mathematical background chapter


\subsubsection{Logical operations, sets, and
functions}\label{logical-operations-sets-and-functions}

These questions assume familiarity with strings, functions, relations,
sets, and logical operators. We use an indexing from zero convention,
and so given a length \(n\) binary string \(x\in \{0,1\}^n\), we denote
coordinates of \(x\) by \(x_0,\ldots,x_{n-1}\). We use \([n]\) to denote
the set \(\{0,1,\ldots,n-1\}\).

\textbf{Question 1 (3 points):} Write a logical expression
\(\varphi(x)\) involving the variables \(x_0,x_1,x_2\) and the operators
\(\wedge\) (AND), \(\vee\) (OR), and \(\neg\) (NOT), such that
\(\varphi(x)\) is true if and only if the majority of the inputs are
\emph{False}.

\textbf{Solution 1:} $\varphi(x):$

$$
(\neg x_0 \land \neg x_1 \land x_2) \lor (\neg x_0 \land x_1 \land \neg x_2)
\lor (x_0 \land \neg x_1 \land \neg x_2) \lor (\neg x_0 \land \neg x_1 \land
\neg x_2)
$$

\textbf{Question 2:} Use the logical quantifiers \(\forall\) (for all),
\(\exists\) (exists), as well as \(\wedge,\vee,\neg\) and the arithmetic
operations \(+,\times,=,>,<\) to write the following:

\textbf{Question 2.1 (3 points):} An expression \(\psi(n,k)\) such that
for every natural numbers \(n,k\), \(\psi(n,k)\) is true if and only if
\(k\) divides \(n\).

\textbf{Solution 2.1:} $\psi(n, k):$

$$
\exists_{c \in \N} (n = ck)
$$

\textbf{Question 2.2 (3 points bonus):} An expression \(\varphi(n)\)
such that for every natural number \(n\), \(\varphi(n)\) is true if and
only if \(n\) is a power of three.

\textbf{Solution 2.2:} $\varphi(n):$

$$
(\forall_{i > 0, i \in \N}\ a_i = 3a_{i-1}) \land (a_0 = 1) \land
(\exists_{a_i}\ n = a_i)
$$

\textbf{Question 3:} In this question, you need to describe in words
sets that are defined using a formula with quantifiers. For example, the
set \(S = \{ x\in \mathbb{N} \;:\; \exists_{y\in\mathbb{N}} x=2y \}\) is
the set of even numbers.

\textbf{Question 3.1 (3 points):} Describe in words the following set
\(S\):

\[
    S = \{ x\in \{0,1\}^{100} : \forall_{i\in \{0,\ldots, 98\}} x_i = x_{i+1} \}
\]

(Recall that, as written in the mathematical background chapter, we use
zero-based indexing in this course, and so a string
\(x\in \{0,1\}^{100}\) is indexed as \(x_0x_1\cdots x_{99}\).)

\textbf{Solution 3.1:} $S$ is the set of binary strings of length 100 with all
bits as either 0 or 1.

\textbf{Question 3.2 (3 points):} Describe in words the following set
\(T\):

\[
    T = \{ x\in \{0,1\}^* : |x|>1 \mbox{ and } \forall_{i \in \{2,\ldots,|x|-1 \} } \forall_{j \in \{2,\ldots,|x|-1\}} i\cdot j \neq |x| \}
\]

\textbf{Solution 3.2:} $T$ is the set of all binary strings of prime length.

\textbf{Question 4:} This question deals with sets, their cardinalities,
and one to one and onto functions. You can cite results connecting these
notions from the course's textbook, MIT's ``Mathematics for Computer
Science'' or any other discrete mathematics textbook.

\textbf{Question 4.1 (4 points):} Define \(S = \{0,1\}^6\) and \(T\) as
the set \(\{ n \in [100] \;|\; \mbox{$n$ is prime } \}\). \emph{Prove or
disprove:} There is a one to one function from \(S\) to \(T\).

\textbf{Solution 4.1:} 
\begin{align*}
    |S| &= 2^{6} = 64\\
    |T| &= 25\\
    \therefore |S| &> |T|
\end{align*}

The problem of mapping elements from $S$ to $T$ can be seen as placing $|S|$
elements in $|T|$ bins. Since $|S| > |T|$ at least one bin will be mapped to
atleast 2 elements. Thus, for any function $f: S \rightarrow T$, there is
atleast one element in $T$ which is mapped to atleast 2 elements in $S$. Hence,
no one-to-one function can exist from $S$ to $T$.

\textbf{Question 4.2 (4 points):} Let \(n=100\),
\(S = [n] \times [n] \times [n]\) and \(T=\{0,1\}^n\). \emph{Prove or
disprove:} There is an onto function from \(T\) to \(S\).

\textbf{Solution 4.2:}
\begin{align*}
    |S| &= n^{3} = 10^{6}\\
    |T| &= 2^{n} = 2^{100}\\
    \therefore |S| &\le |T|
\end{align*}

\begin{lemma}
    \label{th1}
    Given two sets $S$ and $T$ such that, $|T| \le |S|$, There exists an onto 
    function from $S$ to $T$.
\end{lemma}

\textbf{Proof:} Let $S' \subseteq S \mid |S'| = |T|$. \\

Map ${s'}_1 \rightarrow t_1, {s'}_2 \rightarrow t_2, \dots$ where 
${s'}_i \in S'$ and $t_i \in T$. This is a bijective function as every
element in $S'$ is mapped uniquely and every element of $T$ is mapped, since
the size of sets is same.

Map each element in $S\backslash S'$ to any element in $T$. This way, we map $S$ to $T$,
such that the map is onto. Since there exists such a map,
there exists an onto function from $S$ to $T$. 

\qed

By Lemma \ref{th1}, since $|S| \le |T|$, there exists an onto
function from $T$ to $S$.

\textbf{Question 4.3 (4 points):} Let \(n=100\), let
\(S = \{0,1\}^{n^3}\) and \(T\) be the set of all functions mapping
\(\{0,1\}^n\) to \(\{0,1\}\). \emph{Prove or disprove:} There is a one
to one function from \(S\) to \(T\).

\textbf{Solution 4.3:}



\begin{lemma}
    \label{th2}
    Given two sets $S$ and $T$ such that, $|S| \le |T|$, There exists a
    one-to-one function from $S$ to $T$.
\end{lemma}

\textbf{Proof:} Let $T' \subseteq T \mid |T'| = |S|$. \\

Map $s_1 \rightarrow {t'}_1, s_2 \rightarrow {t'}_2, \dots$ where 
$s_i \in S$ and ${t'}_i \in T'$. This is a bijective function as every
element in $S$ is mapped uniquely and every element of $T'$ is mapped, since
the size of sets is same.

Since every element of $S$ is uniquely mapped to an element in $T' \subseteq T$
such that the map is one-to-one we can extend this to a one-to-one
function $f: S \rightarrow T$. Hence, there exists a function from $S$ to $T$
such that it is one-to-one.

\qed

\textbf{Question 5.1 (5 points):} Prove that for every finite sets
\(A,B,C\), \(|A \cup B \cup C| \leq |A|+|B|+|C|\).

\textbf{Solution 5.1:} 
To prove this inequality, we use Inclusion/Exclusion Principle

$$
    |A \cup B| \le |A| + |B|\\
$$

Using the the fact that $|(A \cup B) \cap C| \ge 0$,

$$
    |A \cup B| - |(A \cup B) \cap C| \le |A \cup B|
$$

Using the above two inequalties,

$$
|A \cup B| - |(A \cup B) \cap C| \le |A| + |B|
$$

Adding $|C|$ on both sides,

$$
|A \cup B| + |C| - |(A \cup B) \cap C| \le |A| + |B| + |C|
$$

Using the Inclusion/Exclusion Principle $|A \cup B| = |A| + |B| + |A \cap B|$,

$$
|A \cup B \cup C| \le |A| + |B| + |C|
$$

Hence proved.

\textbf{Question 5.2 (5 points bonus):} Prove that for every finite sets
\(A,B,C\),
\(|A \cup B \cup C| \geq |A|+|B|+|C| - |A \cap B| - |B \cap C| - |A \cap C|\).

\textbf{Solution 5.2:}

Using Inclusion/Exclusion Principle,

$$
|A \cup B \cup C| = |A| + |B| + |C| - |A \cap B| - |B \cap C| - |A \cap C|
+ |A \cap B \cap C|
$$

Since, $|A \cap B \cap C| \ge 0$,

$$
|A \cup B \cup C| \ge |A| + |B| + |C| - |A \cap B| - |B \cap C| - |A \cap C|
$$

Hence Proved.

\subsubsection{Graphs}\label{graphs}

Thee following two questions assume familiarity with basic graph theory.
If you need to look up or review any terms, the CS 121 background page at
{\tt https://cs121.boazbarak.org/background/}
contains several freely available online resources on graph theory. This
material also appears in Chapters 13,14,16 and 17 of the CS 20 textbook
``Essential Discrete Mathematics for Computer Science'' by Harry Lewis
and Rachel Zax.

\textbf{Question 6.1 (5 points):} Prove that if \(G\) is a directed
acyclic graph (DAG) on \(n\) vertices, if \(u\) and \(v\) are two
vertices of \(G\) such that there is a directed path of length \(n-1\)
from \(u\) to \(v\) then \(u\) has no in-neighbors.\footnote{\emph{Hint:}
You can use the topological sorting theorem shown in the mathematical
background chapter.}

\textbf{Solution 6.1:}

Lets assume $\exists_{u^{\prime} \in V}\ u^{\prime} \mid
(u^{\prime}, u) \in E$, where $V$ is the set of all vertices in $G$ 
and $E$ is the set of all edges in $G$.

Since there is a path $\{u^{\prime}, u\}$ of length 1 and a path 
$\{u, u_0, u_1, \dots, v\}$ for $u_i \in V$ of length $n - 1$, there exists
a path $P = \{u^{\prime}, u, u_0, u_1, \dots, v\}$ of length n.\\

In a path of length $n$, there are $n + 1$ vertices. Since there are $n + 1$
vertices in a set and only $n$ total vertices, by pigeonhole principle, atleast
1 vertex is repeated.\\

Since, at least 1 vertex is repeated, there exists a cycle in the path. But by
defination of DAG, there cannot exists a cycle in the graph. This is a
contradiction and hence the assumption that $\exists u^{\prime} \in V$
is wrong.\\

Therefore, indegree of u is zero. Hence, proved.

\textbf{Question 6.2 (5 points):} Prove that for every undirected graph
\(G\) of \(1000\) vertices, if every vertex has degree at most \(4\),
then there exists a subset \(S\) of at least \(200\) vertices such that
no two vertices in \(S\) are neighbors of one another.

\textbf{Solution 6.2:}

Let $V$ denote set of all vertices in $G$ and $U$ denote set of all edges
in $G$.

Let use use the following algorithm to obtain $S$:

\begin{algorithm}
    \caption{Algorithm to obtain $S$}
    \label{calcS}
    \begin{algorithmic}
        \WHILE{$\exists u \in V$}
        \STATE $S \leftarrow S\ \bigcup\ \{u\}$
        \STATE $V \leftarrow V - \{u\}$
        \STATE $V \leftarrow V - \{v \mid (u, v) \in E\}$
        \STATE $E \leftarrow E - \{(u, v) \mid (u, v) \in E\}$
        \ENDWHILE
    \end{algorithmic}
\end{algorithm}

\begin{claim}
    \label{6.2claim}
    Algorithm \ref{calcS} produces a set $S$ such that no two vertices have an
    edge between them in $G$ and is of size atleast $\frac{|V|}{5}$.
\end{claim}

\textbf{Proof:} In any iteration when we choose a vertex,
since we delete all vertices having
an edge to the chosen vertex, Any chosen vertex in further iterations cannot
have an edge to a vertex in $S$. Thus the chosen set $S$ has no two vertices
with an edge between them in $G$.\\

Since, maximum degree of any node is 4, the maximum number
of neighbours it can have is 4. In the worst case, in each iteration, the
chosen vertex is removed from $V$ and at max 4 neighbours are removed. Thus,
in each iteration, at max 5 vertices are removed.\\

Number of iterations $\ge \frac{V}{5}$.\\

In each iteration we add 1 element to $S$.
Therefore $|S|$ = Number of iterations $\ge \frac{|V|}{5}$.\\

\qed

Using Claim \ref{6.2claim},

\begin{align*}
    |S| &\ge \frac{|V|}{5}\\
    |S| &\ge \frac{1000}{5}\\
    |S| &\ge 200
\end{align*}

Hence proved.

\subsubsection{Big-O Notation}\label{big-o-notation}

\textbf{Question 7:} For each pair of functions \(f,g\) below, state
whether or not \(f=O(g)\) and whether or not \(g=O(f)\).

\textbf{Question 7.1 (3 points):} \(f(n)=n(\log n)^3\) and \(g(n)=n^2\).

\textbf{Solution 7.1:} 

$$
f = O(g): \limsup_{n \rightarrow \infty} \frac{f(n)}{g(n)} < \infty
$$

$f = O(g):$

\begin{align*}
    \frac{f(n)}{g(n)} &= \frac{n(\log{n})^{3}}{n^{2}}\\
                      &= \frac{(\log{n})^{2}}{n}\\
    \limsup_{n \rightarrow \infty} \frac{f(n)}{g(n)} &=
    \limsup_{n \rightarrow \infty} \frac{(\log{n})^{2}}{n}\\
    &= 0\\
    &< \infty
\end{align*}

Hence, $f = O(g)$ holds\\

$g = O(f)$

\begin{align*}
    \limsup_{n \rightarrow \infty} \frac{g(n)}{f(n)} &=
    \limsup_{n \rightarrow \infty} \frac{n}{(\log{n})^{2}}\\
    &\rightarrow \infty
\end{align*}

Hence, $g = O(f)$ does not hold.

\textbf{Question 7.2 (3 points):} \(f(n)= n^{\log n}\) and
\(g(n) = n^2\).

\textbf{Solution 7.2:}

$f = O(g):$

\begin{align*}
    \limsup_{n \rightarrow \infty} \frac{f(n)}{g(n)} &=
    \limsup_{n \rightarrow \infty} \frac{n^{\log{n}}}{n^{2}}\\
    &= \frac{n^{\log{n} - 2}}{}\\
    &\rightarrow \infty
\end{align*}

Hence, $f = O(g)$ does not hold.\\

$g = O(f):$

\begin{align*}
    \limsup_{n \rightarrow \infty} \frac{g(n)}{f(n)} &=
    \limsup_{n \rightarrow \infty} n^{2 - \log{n}}\\
    &= 0\\
    &< \infty
\end{align*}

Hence, $g = O(f)$ holds.\\

\textbf{Question 7.3 (3 points bonus):}
\(f(n) = \binom{n}{\lceil 0.2 n \rceil}\) (where \(\binom{n}{k}\) is the
number of \(k\)-sized subsets of a set of size \(n\)) and
\(g(n) = 2^{0.1 n}\).\footnote{\emph{Hint:} one way to do this is to use
{Stirling's approximation} ({\tt https://en.wikipedia.org/wiki/Stirling\%27s\_approximation}).}

\textbf{Solution 7.3:}

\end{document}


