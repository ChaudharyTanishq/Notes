\subsubsection{Background: CPU}

Lets quickly refresh our background on how the CPU works.\\

(fig)\\

\begin{itemize}
    \item Fetch (PC)
    \item Decode: Figure out which instruction it is (inc. PC)
    \item Execture: (could change PC)
\end{itemize}

This view is ok for a single program running. But when we have
multiple programs running, this isnt enough.\\

Lets add more stuff to this:\\

Before execution: Check permission, is this thing OK to execute?\\

If its not ok: Raise an exception $\rightarrow$ OS gets involved 
(exception handler).\\

New instruction after Execution: \textbf{Process Interrupts}. 
Between instructions, there are external events outside of the
program which the system needs to know about. A key has been pressed
, the mouse has been moved, a timer has gone off, etc. These are
things outside the program which need to be handeled.\\

In this step we do: Handle any pending interrupts $\rightarrow$ 
OS gets involved.

privileged mode $\rightarrow$ kernal mode\\
unprivileged mode $\rightarrow$ user mode\\

@boot: OS $\rightarrow$ install "handlers" (code). (Tell h/w what
code to run on exception/interrupts/traps) [done by a privileged
instruction x86: lidt].\\

Another thing OS has to start running a \textbf{timer interrupt}.
[privileged]\\

Now we are ready to run user programs.\\

Timeline:

(fig2 [timeline])\\

\begin{itemize}
    \item "A"" wants OS service (system call)
        Issue special instructions: \textbf{trap} instruction
        (x86 called "int")

    \item    Trap: User mode $\rightarrow$ kernal mode\\
        Jump into OS: target is trap handlers

    \item   Save enough register state so as to enable (so as to 
        enable resume exec. later).

    \item   OS sys call handler runs

    \item  Ret-from-trap
\end{itemize}

This is so much work. This is why jumping into a system call
is much more expensive than a function call.

(fig3)

Timer interrupt: Way to regain control of CPU (by OS). \\

OS handler runs. Can swith to a different process if we want to.
This is called a \textbf{Context switch}
