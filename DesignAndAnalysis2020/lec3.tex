\chapter{What is Computation}

\section{The Turing Machine}

\subsection{Attributes of \textbf{Machine}}

\begin{itemize}
    \item Machines are not Omnipresent\\
        It takes non-zero time to retrieve data from far-off locations\\

        (fig)\\

        "Very Large" memories are inherently \textbf{sequential}.
    \item Machines are not Omniscient\\
        Only finite information can be stores/retrieved from finite volume\\

        (fig)\\

        \textbf{Finite set} of tape symbols.
    \item Machines are not Omnipotent\\
        A finite length code only exerts finite amount of control\\

        (fig)
\end{itemize}

\subsection{Turing Machine Defination}

The \textit{Simplest} Programming Language\\

\begin{itemize}
    \item TM is a 7-tuple (Q, sigma, r, delta, $q_0$, $q_{accept}$,
        $q_{reject}$)
        \begin{itemize}
            \item Q is the finite set of states
            \item sigma is the finite input alphabest set, not containing
                blank
            \item r is the finite tape alphabet, with blank
            \item delta: (Q $\times$ R) to (Q $\timex$ r $\timex$ )
        \end{itemize}
\end{itemize}

\textbf{C-Programs Are Turing Complete}\\

No C-program exists Implies No Algorithm/Machine exists!\\

\section{Recognizing Versus Deciding Languages}

A TM M \textbf{recognizes} language L if:

For all $w$ in $L$, $M$ accepts $w$\\
For all $w$ not in $L$, $M$ does not accept $w$\\

A TM M \textbf{decides} language L if:

For all $w$ in $L$, $M$ accepts $w$\\
For all $w$ in $L$m $M$ rejects $w$\\

There is difference in \textit{rejects} and \textit{does not accept}. 

\section{Decidability and Recognizability}

\begin{itemize}
    \item L is decidable if some TM M decides L
    \item L is recognizable if some TM M recognizes L
    \item Recursive Class (R): The set of all decidable languages
    \item Recursively Enumerable Class (RE): The set of all recognizable 
        languages
    \item 
\end{itemize}

\subsection{Languages that are recognizable but not decidable}


