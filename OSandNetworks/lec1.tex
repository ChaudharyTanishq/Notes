\chapter{Lecture 1}

\section{Outline}

\begin{itemize}
    \item What is an Operating System
    \item Course topics and grading
    \item History, development and concepts of Oss
    \item Different kinds of Computer Systems
    \item Concept of virtual computer
\end{itemize}

\section{What is a system?}

A system is an inter-related set of components with an identifiable
boundary working together for some purpose.\\

System can be natural or fabricated:
\begin{itemize}
    \item Natural systems: human body or solar system
    \item Fabricated systems: cycle, bus, etc.
\end{itemize}

\subsection{Characteristics of a system}

\begin{itemize}
    \item Components:
        \begin{itemize}
            \item A system is made up of components
            \item A compenent is either an irreducible part or aggregation of
                parts that make-up a system. A component is also called a
                sub-system.
        \end{itemize}

    \item Interrelated:
        \begin{itemize}
            \item The components of interrelated
            \item Dependence of one subsystem on one or more subsystems.
        \end{itemize}

    \item Boundary:
        \begin{itemize}
            \item A system has a boundary, within which all of its components
                are contained and which establishes the lmits of a separating
                the system from other systems.
        \end{itemize}

    \item Purpose:
        \begin{itemize}
            \item The overall goal of a function of a system.
            \item The system's reason for existing.
            \item If you want to design any system, first you need a purpose.
        \end{itemize}

    \item Environment:
        \begin{itemize}
            \item Everythin external to the system that interacts with the 
                system.
        \end{itemize}

    \item Interface:
        \begin{itemize}
            \item Point of contact where a system meets its environment
                ot subsystems meet each other.
        \end{itemize}

    \item Constraint:
        \begin{itemize}
            \item A limit what a system can accomplish:
                Capacity, speed or capabilities.
        \end{itemize}

    \item Input: 
        \begin{itemize}
            \item Whatever a system takes from its environment in order
                to fulfull its purpose.
        \end{itemize}

    \item Output:
        \begin{itemize}
            \item Whatever a system returns to the envrionment.
        \end{itemize}
\end{itemize}

\section{Important System Concepts}

\subsection{Decomposition}

\begin{itemize}
    \item It deals with being able to break down a system into
        its components.
    \item Decomposition results in smaller and less complex pieces
        which are easier to conqour.
    \item Decomposing a system allows to focus on a particular 
        part of system.
\end{itemize}

\subsection{Modulariy}

\begin{itemize}
    \item Dividing a system into chunks or modules of uniform size.
    \item Can edit, replace or add another module without
        effecting the rest of the system.
    \item Helps in reducing dependencies between systems.
\end{itemize}

\subsection{Conesion}

\begin{itemize}
    \item TODO 
\end{itemize}

\section{What is an operating system?}

Operating system is a subsystem of a tool which faciliates the
operation of the tool.\\

For a user, the operating system abstracts the machine part
in terms of simple services by hiding the details of the machine.
The OS can provide services to users or other subsystems.

\section{What is a computer operating system?}

A computer is also a tool that contains a machine part and an
operating part. For a computer, the operating system abstracts
the underlying hardware in terms of simple services by hiding
the details of the hardware.\\

For the rest of the course, operating system refers to computer
operating system.\\

The operating system sits on top of the hardware interacts with
the system and application programs. You cannot interact with
the hardware directly. You have to interact with the OS to interact
with the hardware.\\

\paragraph{Book Defination}
A program that acts as an intermediary between a user of a 
computer and the computer hardware.

\paragraph{Operating system goals}
Make the computer system convenient to use, Use the hardware in
an efficent way.

\section{OS Definitions}

\begin{itemize}
    \item Resource Allocator: Manages and allocates resources.
    \item Control Program: Controls the execution of user programs
        and operations of I/O devices.
    \item Kernal: The one program running all the time.
\end{itemize}
